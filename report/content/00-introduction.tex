\section{Introduction}

The process of water treatment involves the elimination of any biological,
chemical, or physical elements that could potentially endanger the safety of the
water supply intended for human and household consumption.

The process aids in generating water that is free from harmful substances,
pleasant-tasting, transparent, lacking in color and smell, and also
non-corrosive so as not to harm pipelines. \cite{TheConstructor}

\subsection{Coagulation}

Coagulation is a process that eliminates tiny particles (smaller than 1 µm) that
are floating in the water.

A coagulant, possessing a positive electrical charge, is included in the water
to counteract the negative electrical charge of the fine particles in this
procedure.

Once their charges are neutralized, the fine particles merge to create soft and
light clusters known as ``flocs.''

Aluminum sulfate and ferric chloride are the two most commonly used coagulants
in water treatment.

\subsection{Flocculation}

In this process, paddles softly agitate the water within a flocculation basin,
causing flocs to combine and create bigger flocs.
